% Options for packages loaded elsewhere
\PassOptionsToPackage{unicode}{hyperref}
\PassOptionsToPackage{hyphens}{url}
%
\documentclass[
]{article}
\usepackage{amsmath,amssymb}
\usepackage{iftex}
\ifPDFTeX
  \usepackage[T1]{fontenc}
  \usepackage[utf8]{inputenc}
  \usepackage{textcomp} % provide euro and other symbols
\else % if luatex or xetex
  \usepackage{unicode-math} % this also loads fontspec
  \defaultfontfeatures{Scale=MatchLowercase}
  \defaultfontfeatures[\rmfamily]{Ligatures=TeX,Scale=1}
\fi
\usepackage{lmodern}
\ifPDFTeX\else
  % xetex/luatex font selection
\fi
% Use upquote if available, for straight quotes in verbatim environments
\IfFileExists{upquote.sty}{\usepackage{upquote}}{}
\IfFileExists{microtype.sty}{% use microtype if available
  \usepackage[]{microtype}
  \UseMicrotypeSet[protrusion]{basicmath} % disable protrusion for tt fonts
}{}
\makeatletter
\@ifundefined{KOMAClassName}{% if non-KOMA class
  \IfFileExists{parskip.sty}{%
    \usepackage{parskip}
  }{% else
    \setlength{\parindent}{0pt}
    \setlength{\parskip}{6pt plus 2pt minus 1pt}}
}{% if KOMA class
  \KOMAoptions{parskip=half}}
\makeatother
\usepackage{xcolor}
\usepackage[margin=1in]{geometry}
\usepackage{color}
\usepackage{fancyvrb}
\newcommand{\VerbBar}{|}
\newcommand{\VERB}{\Verb[commandchars=\\\{\}]}
\DefineVerbatimEnvironment{Highlighting}{Verbatim}{commandchars=\\\{\}}
% Add ',fontsize=\small' for more characters per line
\usepackage{framed}
\definecolor{shadecolor}{RGB}{248,248,248}
\newenvironment{Shaded}{\begin{snugshade}}{\end{snugshade}}
\newcommand{\AlertTok}[1]{\textcolor[rgb]{0.94,0.16,0.16}{#1}}
\newcommand{\AnnotationTok}[1]{\textcolor[rgb]{0.56,0.35,0.01}{\textbf{\textit{#1}}}}
\newcommand{\AttributeTok}[1]{\textcolor[rgb]{0.77,0.63,0.00}{#1}}
\newcommand{\BaseNTok}[1]{\textcolor[rgb]{0.00,0.00,0.81}{#1}}
\newcommand{\BuiltInTok}[1]{#1}
\newcommand{\CharTok}[1]{\textcolor[rgb]{0.31,0.60,0.02}{#1}}
\newcommand{\CommentTok}[1]{\textcolor[rgb]{0.56,0.35,0.01}{\textit{#1}}}
\newcommand{\CommentVarTok}[1]{\textcolor[rgb]{0.56,0.35,0.01}{\textbf{\textit{#1}}}}
\newcommand{\ConstantTok}[1]{\textcolor[rgb]{0.00,0.00,0.00}{#1}}
\newcommand{\ControlFlowTok}[1]{\textcolor[rgb]{0.13,0.29,0.53}{\textbf{#1}}}
\newcommand{\DataTypeTok}[1]{\textcolor[rgb]{0.13,0.29,0.53}{#1}}
\newcommand{\DecValTok}[1]{\textcolor[rgb]{0.00,0.00,0.81}{#1}}
\newcommand{\DocumentationTok}[1]{\textcolor[rgb]{0.56,0.35,0.01}{\textbf{\textit{#1}}}}
\newcommand{\ErrorTok}[1]{\textcolor[rgb]{0.64,0.00,0.00}{\textbf{#1}}}
\newcommand{\ExtensionTok}[1]{#1}
\newcommand{\FloatTok}[1]{\textcolor[rgb]{0.00,0.00,0.81}{#1}}
\newcommand{\FunctionTok}[1]{\textcolor[rgb]{0.00,0.00,0.00}{#1}}
\newcommand{\ImportTok}[1]{#1}
\newcommand{\InformationTok}[1]{\textcolor[rgb]{0.56,0.35,0.01}{\textbf{\textit{#1}}}}
\newcommand{\KeywordTok}[1]{\textcolor[rgb]{0.13,0.29,0.53}{\textbf{#1}}}
\newcommand{\NormalTok}[1]{#1}
\newcommand{\OperatorTok}[1]{\textcolor[rgb]{0.81,0.36,0.00}{\textbf{#1}}}
\newcommand{\OtherTok}[1]{\textcolor[rgb]{0.56,0.35,0.01}{#1}}
\newcommand{\PreprocessorTok}[1]{\textcolor[rgb]{0.56,0.35,0.01}{\textit{#1}}}
\newcommand{\RegionMarkerTok}[1]{#1}
\newcommand{\SpecialCharTok}[1]{\textcolor[rgb]{0.00,0.00,0.00}{#1}}
\newcommand{\SpecialStringTok}[1]{\textcolor[rgb]{0.31,0.60,0.02}{#1}}
\newcommand{\StringTok}[1]{\textcolor[rgb]{0.31,0.60,0.02}{#1}}
\newcommand{\VariableTok}[1]{\textcolor[rgb]{0.00,0.00,0.00}{#1}}
\newcommand{\VerbatimStringTok}[1]{\textcolor[rgb]{0.31,0.60,0.02}{#1}}
\newcommand{\WarningTok}[1]{\textcolor[rgb]{0.56,0.35,0.01}{\textbf{\textit{#1}}}}
\usepackage{graphicx}
\makeatletter
\def\maxwidth{\ifdim\Gin@nat@width>\linewidth\linewidth\else\Gin@nat@width\fi}
\def\maxheight{\ifdim\Gin@nat@height>\textheight\textheight\else\Gin@nat@height\fi}
\makeatother
% Scale images if necessary, so that they will not overflow the page
% margins by default, and it is still possible to overwrite the defaults
% using explicit options in \includegraphics[width, height, ...]{}
\setkeys{Gin}{width=\maxwidth,height=\maxheight,keepaspectratio}
% Set default figure placement to htbp
\makeatletter
\def\fps@figure{htbp}
\makeatother
\setlength{\emergencystretch}{3em} % prevent overfull lines
\providecommand{\tightlist}{%
  \setlength{\itemsep}{0pt}\setlength{\parskip}{0pt}}
\setcounter{secnumdepth}{-\maxdimen} % remove section numbering
\ifLuaTeX
  \usepackage{selnolig}  % disable illegal ligatures
\fi
\IfFileExists{bookmark.sty}{\usepackage{bookmark}}{\usepackage{hyperref}}
\IfFileExists{xurl.sty}{\usepackage{xurl}}{} % add URL line breaks if available
\urlstyle{same}
\hypersetup{
  pdftitle={Introduction to the course},
  hidelinks,
  pdfcreator={LaTeX via pandoc}}

\title{Introduction to the course}
\author{}
\date{\vspace{-2.5em}}

\begin{document}
\maketitle

\begin{center}\rule{0.5\linewidth}{0.5pt}\end{center}

\hypertarget{introductions}{%
\subsubsection{Introductions}\label{introductions}}

I can start this off. Then we can go around the room, thinking about

\begin{itemize}
\tightlist
\item
  what you want to get out of the course
\item
  how theory fits into your current research (if you are currently doing
  research)
\item
  any worries/feelings/questions/concerns about the course
\end{itemize}

\begin{center}\rule{0.5\linewidth}{0.5pt}\end{center}

\hypertarget{syllabus}{%
\subsubsection{Syllabus}\label{syllabus}}

Work through syllabus, clarify structure, discuss final project/Case
text/homework structure/paper summaries/etc.

\begin{itemize}
\tightlist
\item
  \href{https://theoreticalEcology.github.io/syllabus/}{syllabus}
\end{itemize}

\begin{center}\rule{0.5\linewidth}{0.5pt}\end{center}

\hypertarget{course-website}{%
\subsubsection{Course website}\label{course-website}}

Go through the course website and make sure everyone has easy access and
is clear on how to get the lecture notes, readings, and homeworks.

\begin{center}\rule{0.5\linewidth}{0.5pt}\end{center}

\hypertarget{notes}{%
\subsubsection{Notes}\label{notes}}

\begin{itemize}
\tightlist
\item
  You may want to \href{https://www.r-project.org/}{download R} as we
  will be using it to simulate some of the models we go over. However,
  we will not be learning how to use R during the course (as this would
  be a bit much to learn at the same time as the theory), but I
  definitely encourage folks to check out the resources available for
  working in R, as scientific research tends to need a bit of
  programming.
\end{itemize}

\begin{center}\rule{0.5\linewidth}{0.5pt}\end{center}

\hypertarget{what-is-theoretical-ecology}{%
\subsubsection{What is theoretical
ecology?}\label{what-is-theoretical-ecology}}

\textbf{Ecology}: study of species interactions

\textbf{Theory}: Body of evidence linking a general explanation to a
commonly observed phenomenon

\begin{quote}
Theoretical ecology attempts to explain patterns observed in nature
through generalized models
\end{quote}

\begin{center}\rule{0.5\linewidth}{0.5pt}\end{center}

\hypertarget{models}{%
\subsubsection{Models}\label{models}}

\textbf{Statistical model}: A model exploring the statistical
relationships between a response and some set of predictor variables.

\begin{itemize}
\item
  can be completely independent of theory
\item
  can be focused on prediction or variable importance
\item
  requires data
\end{itemize}

e.g., linear regression

\begin{center}\rule{0.5\linewidth}{0.5pt}\end{center}

\hypertarget{what-is-the-relationship-betwen-sepal-length-and-sepal-length-in-irises}{%
\subsubsection{What is the relationship betwen sepal length and sepal
length in
irises?}\label{what-is-the-relationship-betwen-sepal-length-and-sepal-length-in-irises}}

\begin{Shaded}
\begin{Highlighting}[]
\FunctionTok{data}\NormalTok{(iris)}
\NormalTok{mod }\OtherTok{\textless{}{-}} \FunctionTok{glm}\NormalTok{(iris}\SpecialCharTok{$}\NormalTok{Sepal.Length }\SpecialCharTok{\textasciitilde{}}\NormalTok{ iris}\SpecialCharTok{$}\NormalTok{Sepal.Width)}
\FunctionTok{plot}\NormalTok{(iris}\SpecialCharTok{$}\NormalTok{Sepal.Length }\SpecialCharTok{\textasciitilde{}}\NormalTok{ iris}\SpecialCharTok{$}\NormalTok{Sepal.Width, }
  \AttributeTok{col=}\DecValTok{1}\SpecialCharTok{:}\DecValTok{3}\NormalTok{[}\FunctionTok{as.numeric}\NormalTok{(iris}\SpecialCharTok{$}\NormalTok{Species)])}
\FunctionTok{abline}\NormalTok{(mod)}
\end{Highlighting}
\end{Shaded}

\includegraphics{/media/tad/sanDisk1TB/Teaching/theoreticalEcology/website/lecturePDF/setup_files/figure-latex/unnamed-chunk-1-1.pdf}

\begin{center}\rule{0.5\linewidth}{0.5pt}\end{center}

\textbf{Phenomenological model}: A model describing some phenomenon,
often independent of any data.

\begin{itemize}
\item
  an idea of \emph{how} a system behaves given assumptions
\item
  incorporates theoretical expectations
\item
  incredibly useful when fit or compared to empirical data (where's the
  disconnect?)
\end{itemize}

e.g., exponential growth

\begin{center}\rule{0.5\linewidth}{0.5pt}\end{center}

\hypertarget{population-dynamics-with-exponential-growth}{%
\subsubsection{Population dynamics with exponential
growth}\label{population-dynamics-with-exponential-growth}}

\[ N_{t+1} = N_t \lambda \]

\begin{Shaded}
\begin{Highlighting}[]
\NormalTok{expoGrowth }\OtherTok{\textless{}{-}} \ControlFlowTok{function}\NormalTok{(n, }\AttributeTok{lambda=}\FloatTok{1.25}\NormalTok{, }\AttributeTok{times=}\DecValTok{100}\NormalTok{)\{}
\NormalTok{  nt }\OtherTok{\textless{}{-}} \FunctionTok{c}\NormalTok{(n)}
  \ControlFlowTok{for}\NormalTok{(i }\ControlFlowTok{in} \DecValTok{1}\SpecialCharTok{:}\NormalTok{times)\{}
\NormalTok{    nt[i}\SpecialCharTok{+}\DecValTok{1}\NormalTok{] }\OtherTok{\textless{}{-}}\NormalTok{ nt[i] }\SpecialCharTok{*}\NormalTok{ lambda}
\NormalTok{  \}}
  \FunctionTok{return}\NormalTok{(nt)}
\NormalTok{\}}
\end{Highlighting}
\end{Shaded}

\begin{center}\rule{0.5\linewidth}{0.5pt}\end{center}

\begin{Shaded}
\begin{Highlighting}[]
\FunctionTok{plot}\NormalTok{(}\FunctionTok{expoGrowth}\NormalTok{(}\DecValTok{10}\NormalTok{), }\AttributeTok{type=}\StringTok{\textquotesingle{}l\textquotesingle{}}\NormalTok{, }\AttributeTok{ylab=}\StringTok{\textquotesingle{}Population size\textquotesingle{}}\NormalTok{, }
  \AttributeTok{xlab=}\StringTok{\textquotesingle{}Time\textquotesingle{}}\NormalTok{, }\AttributeTok{lwd=}\DecValTok{2}\NormalTok{)}
\FunctionTok{lines}\NormalTok{(}\FunctionTok{expoGrowth}\NormalTok{(}\DecValTok{10}\NormalTok{, }\AttributeTok{lambda=}\DecValTok{2}\NormalTok{), }\AttributeTok{col=}\StringTok{\textquotesingle{}dodgerblue\textquotesingle{}}\NormalTok{, }\AttributeTok{lwd=}\DecValTok{2}\NormalTok{)}
\end{Highlighting}
\end{Shaded}

\includegraphics{/media/tad/sanDisk1TB/Teaching/theoreticalEcology/website/lecturePDF/setup_files/figure-latex/unnamed-chunk-3-1.pdf}

\begin{center}\rule{0.5\linewidth}{0.5pt}\end{center}

\begin{quote}
This course will focus pretty much entirely on phenomenological models
(as I have defined them)
\end{quote}

\begin{quote}
Also, evolutionary models do not really get emphasized in this book, but
this could be good fodder for your paper summaries or final projects if
you are interested
\end{quote}

\begin{center}\rule{0.5\linewidth}{0.5pt}\end{center}

\hypertarget{why-theoretical-ecology}{%
\subsubsection{Why theoretical ecology?}\label{why-theoretical-ecology}}

\begin{itemize}
\tightlist
\item
  Theory allows us a lens with which to explain natural systems.
\end{itemize}

\textbf{Observation}: some set of ecological processes is causing this
population to cycle

\textbf{Conclusion}: this population is cycling

\textbf{Theory}: which of the cycle-generating processes could cause
cycles?

\textbf{Generation of novel question}: can models create cycles under
processes with different `strengths'?

\begin{center}\rule{0.5\linewidth}{0.5pt}\end{center}

\hypertarget{why-theoretical-ecology-1}{%
\subsubsection{Why theoretical
ecology?}\label{why-theoretical-ecology-1}}

\begin{itemize}
\tightlist
\item
  It provides a useful interface for experimental and observational
  studies.
\end{itemize}

\textbf{Population model}: Ricker model to explore population dynamics
and the role of demographic/environmental stochasticity

\textbf{Experiment}: fit models to experimental data and find
demographic stochasticity important in controlled laboratory experiments

\textbf{Observation}: fit models to empirical time series to explore the
relative importance of demographic and environmental stochasticity in
natural systems

\begin{center}\rule{0.5\linewidth}{0.5pt}\end{center}

\hypertarget{why-theoretical-ecology-2}{%
\subsubsection{Why theoretical
ecology?}\label{why-theoretical-ecology-2}}

\begin{itemize}
\tightlist
\item
  It shows us what ecological interactions \emph{could} look like in
  different conditions
\end{itemize}

\includegraphics[width=1\textwidth,height=\textheight]{https://besjournals.onlinelibrary.wiley.com/cms/asset/7c4f8326-a40c-45f3-bb51-9693a5761a6d/jane13485-fig-0003-m.jpg}

\begin{center}\rule{0.5\linewidth}{0.5pt}\end{center}

\hypertarget{why-theoretical-ecology-3}{%
\subsubsection{Why theoretical
ecology?}\label{why-theoretical-ecology-3}}

\begin{itemize}
\tightlist
\item
  It provides us a concrete way to explore and explain natural processes
\end{itemize}

\includegraphics[width=0.6\textwidth,height=\textheight]{https://www.geographyrealm.com/wp-content/uploads/2014/11/Island-biogeography.jpg}

\begin{center}\rule{0.5\linewidth}{0.5pt}\end{center}

\hypertarget{barriers-to-a-more-theory-rich-discipline}{%
\subsubsection{Barriers to a more theory-rich
discipline}\label{barriers-to-a-more-theory-rich-discipline}}

\begin{itemize}
\tightlist
\item
  some ecologists are `bad' at theory. Why? (math/programming/courses)
\item
  Is the world too complex to model? No (most of the time)

  \begin{itemize}
  \tightlist
  \item
    Is a complex model useful? No (most of the time)
  \end{itemize}
\end{itemize}

\url{https://www.cell.com/trends/ecology-evolution/fulltext/S0169-5347(19)30171-5}

\begin{center}\rule{0.5\linewidth}{0.5pt}\end{center}

\hypertarget{the-promise-of-theory-in-ecology}{%
\subsubsection{The promise of theory in
ecology}\label{the-promise-of-theory-in-ecology}}

\begin{itemize}
\tightlist
\item
  we have lots of patterns, perhaps more patterns than theory?
\end{itemize}

\begin{quote}
`To do science is to search for repeated patterns\ldots{} The best
person to do this {[}in ecology{]} is the naturalist who loves to note
changes in bird life up a mountainside, or changes in plant life from
mainland to island, or changes in butterflies from temperate to tropics'
- Robert MacArthur
\end{quote}

\emph{note: I 100\% disagree with the assertion that there is a `best'
type of ecologist, and that the `best' type would be a naturalist}

\begin{center}\rule{0.5\linewidth}{0.5pt}\end{center}

\hypertarget{patterns-without-theory-can-be-deceiving}{%
\subsubsection{Patterns without theory can be
deceiving}\label{patterns-without-theory-can-be-deceiving}}

\begin{figure}
\centering
\includegraphics[width=0.7\textwidth,height=\textheight]{https://esajournals.onlinelibrary.wiley.com/cms/asset/edd898a5-4c22-468b-83d5-b78460ac7f97/ecy201596123386-fig-0001-m.jpg}
\caption{distribution of birds in Bismarck Archipelago}
\end{figure}

\url{https://esajournals.onlinelibrary.wiley.com/doi/10.1890/14-1848.1}

\begin{quote}
Where did Diamond go wrong?
\end{quote}

\begin{center}\rule{0.5\linewidth}{0.5pt}\end{center}

\hypertarget{theory-requires-tests}{%
\subsubsection{Theory requires tests}\label{theory-requires-tests}}

There is an inherent feedback in developing some conceptual theory based
on observations, and then testing this theory in different locations or
in experimental trials.

\includegraphics[width=0.3\textwidth,height=\textheight]{https://saylordotorg.github.io/text_general-chemistry-principles-patterns-and-applications-v1.0/section_05/4100bb1759822739546b1e01c77733a3.jpg}

\begin{center}\rule{0.5\linewidth}{0.5pt}\end{center}

\hypertarget{breaking-the-training-divide}{%
\subsubsection{Breaking the training
divide}\label{breaking-the-training-divide}}

\includegraphics[width=0.6\textwidth,height=\textheight]{https://www.cell.com/cms/attachment/961bdd89-cb83-4b26-9b7b-d409690c7b1d/gr2.jpg}

\url{https://www.cell.com/trends/ecology-evolution/fulltext/S0169-5347(19)30171-5}

\begin{center}\rule{0.5\linewidth}{0.5pt}\end{center}

\hypertarget{so-youre-like-a-modeler}{%
\subsubsection{So, you're like a
`modeler'?}\label{so-youre-like-a-modeler}}

\begin{itemize}
\tightlist
\item
  The `modeler'/`empiricist' divide is a misnomer and is silly
\item
  The goal of this course is not to pull you away from natural systems,
  but to give you tools to generalize and explain the dynamics of
  natural systems
\end{itemize}

\url{https://royalsocietypublishing.org/doi/10.1098/rsfs.2012.0008}

\url{https://www.journals.uchicago.edu/doi/pdfplus/10.1086/717206}

\begin{center}\rule{0.5\linewidth}{0.5pt}\end{center}

\hypertarget{discussion}{%
\subsubsection{Discussion}\label{discussion}}

\begin{quote}
Codling and Dumbrell. 2012. Mathematical and theoretical ecology:linking
models with ecological processes. \emph{Interface}
\url{doi:10.1098/rsfs.2012.0008}
\end{quote}

\textbf{Note}:

\begin{itemize}
\item
  This is an introduction to a special issue organized around
  `Mathematical and theoretical ecology', so it spends some time trying
  to tie the papers into a cohesive picture. This is useful for this
  course, as it gives you an idea of types of theoretical articles.
\item
  \href{http://faculty.washington.edu/cet6/pub/Temp/CFR521e/Lawton_1996.pdf}{Additional
  article some folks might find interesting}: Lawton discussing
  `patterns' back in 1996, muddying some waters while also having some
  pretty fair points in other places.
\end{itemize}

\begin{center}\rule{0.5\linewidth}{0.5pt}\end{center}

\hypertarget{the-paper-was-written-over-a-decade-ago.-in-what-ways-do-you-think-things-have-changed-in-the-field}{%
\subsubsection{The paper was written over a decade ago. In what ways do
you think things have changed in the
field?}\label{the-paper-was-written-over-a-decade-ago.-in-what-ways-do-you-think-things-have-changed-in-the-field}}

\begin{center}\rule{0.5\linewidth}{0.5pt}\end{center}

\hypertarget{what-were-your-reactions-when-you-read-this}{%
\subsubsection{What were your reactions when you read
this?}\label{what-were-your-reactions-when-you-read-this}}

\begin{quote}
``Without ecological theory, collecting data is a futile and meaningless
endeavour. Likewise, producing elegantand beautiful mathematical models
of ecological systems without validation against real data is an empty
achievement''.
\end{quote}

\begin{center}\rule{0.5\linewidth}{0.5pt}\end{center}

\hypertarget{simplifying-assumptions-and-the-mean-field-approximation}{%
\subsubsection{Simplifying assumptions and the mean-field
approximation}\label{simplifying-assumptions-and-the-mean-field-approximation}}

\begin{itemize}
\tightlist
\item
  The authors discuss (bottom left of page 145) the use of mean-field
  techniques and sometimes when assumptions used in theoretical models
  may be alright. This is an important point, because we will make
  oodles of simplifying assumptions. What are some of the tradeoffs in
  using simplifying assumptions (both from the article and in your own
  research)?
\end{itemize}

\begin{center}\rule{0.5\linewidth}{0.5pt}\end{center}

\hypertarget{generating-testable-predictions}{%
\subsubsection{Generating testable
predictions}\label{generating-testable-predictions}}

\begin{itemize}
\tightlist
\item
  One role of theory is to generate testable hypotheses. The Rands paper
  from this issue (discussed on page 147) is a great example. Can we
  collectively think of other examples of how theoretical models, in the
  absence of data, have generated testable predictions?
\end{itemize}

\begin{center}\rule{0.5\linewidth}{0.5pt}\end{center}

\hypertarget{lines-in-the-sand}{%
\subsubsection{Lines in the sand}\label{lines-in-the-sand}}

\begin{quote}
Broadly speaking, ecological models can be split into two separate
categories; simplistic mathematical models, which offer analytically
tractable solutions and the examination of the underlying model
properties, and more complex simulation-based models (cannot be solved
analytically)
\end{quote}

\begin{itemize}
\tightlist
\item
  What does it mean to `analytically solve' a model, and how is this
  useful?
\item
  To what extent has this distinction been eroded in more recent years
  and why/how?
\end{itemize}

\begin{center}\rule{0.5\linewidth}{0.5pt}\end{center}

\hypertarget{role-of-stochastic-processes}{%
\subsubsection{Role of stochastic
processes}\label{role-of-stochastic-processes}}

\begin{quote}
Ecological systems are very different from physical and chemical
systems, notably containing more uncertainty and chaotic dynamics, and
are often influenced by stochastic processes.
\end{quote}

\begin{itemize}
\tightlist
\item
  This gets at two things

  \begin{enumerate}
  \def\labelenumi{\arabic{enumi}.}
  \tightlist
  \item
    ecological data are messy (imperfect detection etc.)
  \item
    ecological systems are messy (stochastic birth/death)
  \end{enumerate}
\end{itemize}

To what extent can models incorporate either/or of these processes,
allowing theory to easily bridge this data-model divide the authors try
to emphasize?

\begin{center}\rule{0.5\linewidth}{0.5pt}\end{center}

\hypertarget{ecology-in-the-absence-of-theory}{%
\subsubsection{Ecology in the absence of
theory}\label{ecology-in-the-absence-of-theory}}

\begin{quote}
After all, without theory providing testable hypotheses, ecology could
become nothing more than data collection for its own sake
\end{quote}

\begin{itemize}
\tightlist
\item
  The authors are taking a bit of a biased (if not obtuse) view of the
  field of ecology here (in my opinion). I was curious if this line
  (towards of the end of article) generated any thoughts from you all.
\end{itemize}

\end{document}
